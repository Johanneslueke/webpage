\usepackage{xpatch}

\setlength\bibhang{1cm}

%%% Weitere Optionen
%\boolitem[false]{citexref} %Wenn incollection, inbook, inproceedings genutzt wird nicht den zugehörigen parent auch in Literaturverzeichnis aufnehmen

%Aufräumen die Felder werden laut Leitfaden nicht benötigt.
\AtEveryBibitem{%
\ifentrytype{book}{
    \clearfield{issn}%
    \clearfield{doi}%
    \clearfield{isbn}%
    \clearfield{url}
    \clearfield{eprint}
}{}
\ifentrytype{collection}{
  \clearfield{issn}%
  \clearfield{doi}%
  \clearfield{isbn}%
  \clearfield{url}
  \clearfield{eprint}
}{}
\ifentrytype{incollection}{
  \clearfield{issn}%
  \clearfield{doi}%
  \clearfield{isbn}%
  \clearfield{url}
  \clearfield{eprint}
}{}
\ifentrytype{article}{
  \clearfield{issn}%
  \clearfield{doi}%
  \clearfield{isbn}%
  \clearfield{url}
  \clearfield{eprint}
}{}
\ifentrytype{inproceedings}{
  \clearfield{issn}%
  \clearfield{doi}%
  \clearfield{isbn}%
  \clearfield{url}
  \clearfield{eprint}
}{}
}

\renewcommand*{\finentrypunct}{}%Kein Punkt am ende des Literaturverzeichnisses

\renewcommand*{\newunitpunct}{\addcomma\space}
\DeclareDelimFormat[bib,biblist]{nametitledelim}{\addcolon\space}
\DeclareDelimFormat{titleyeardelim}{\newunitpunct}
%Namen kursiv schreiben
\renewcommand*{\mkbibnamefamily}{\mkbibemph}
\renewcommand*{\mkbibnamegiven}{\mkbibemph}
\renewcommand*{\mkbibnamesuffix}{\mkbibemph}
\renewcommand*{\mkbibnameprefix}{\mkbibemph}

% Die Trennung mehrerer Autorennamen erfolgt durch Kommata.
% siehe Beispiele im Leitfaden S. 16
% Die folgende Zeile würde mit Semikolon trennen
%\DeclareDelimFormat{multinamedelim}{\addsemicolon\addspace}

%Delimiter für mehrere und letzten Namen gleich setzen
\DeclareDelimAlias{finalnamedelim}{multinamedelim}

\DeclareNameAlias{default}{family-given}
\DeclareNameAlias{sortname}{default}  %Nach Namen sortieren


\DeclareFieldFormat{editortype}{\mkbibparens{#1}}
\DeclareDelimFormat{editortypedelim}{\addspace}
\DeclareFieldFormat{translatortype}{\mkbibparens{#1}}
\DeclareDelimFormat{translatortypedelim}{\addspace}
\DeclareDelimFormat[bib,biblist]{innametitledelim}{\addcomma\space}

\DeclareFieldFormat*{citetitle}{#1}
\DeclareFieldFormat*{title}{#1}
\DeclareFieldFormat*{booktitle}{#1}
\DeclareFieldFormat*{journaltitle}{#1}

\xpatchbibdriver{online}
  {\usebibmacro{organization+location+date}\newunit\newblock}
  {}
  {}{}

\DeclareFieldFormat[online]{date}{\mkbibparens{#1}}
\DeclareFieldFormat{urltime}{\addspace #1\addspace \langde{Uhr}\langen{MEZ}}
\DeclareFieldFormat{urldate}{%urltime zu urldate hinzufügen
  [\langde{Zugriff}\langen{Access}\addcolon\addspace
  #1\printfield{urltime}]
}
\DeclareFieldFormat[online]{url}{<\url{#1}>}
\renewbibmacro*{url+urldate}{%
  \usebibmacro{url}%
  \ifentrytype{online}
    {\setunit*{\addspace}%
     \iffieldundef{year}
       {\printtext[date]{\langde{keine Datumsangabe}\langen{no Date} }}
       {\usebibmacro{date}}}%
    {}%
  \setunit*{\addspace}%
  \usebibmacro{urldate}
  }

%Verhindern, dass bei mehreren Quellen des gleichen Autors im gleichen Jahr
%Buchstaben nach der Jahreszahl angezeigt werden wenn sich das Keyword in usera unterscheidet.
\DeclareExtradate{
  \scope{
    \field{labelyear}
    \field{year}
    }
    \scope{
      \field{usera}
     }
}

%% Anzeige des Jahres nach dem Stichwort (usera) im Literaturverzeichnis
%% Wenn das Jahr bei Online-Quellen nicht explizit angegeben wurde, wird nach
%% dem Stichwort 'o. J.' ausgegeben. Nach der URL steht dann 'keine
%% Datumsangabe'. Ist das Jahr definiert, wird es an beiden Stellen ausgegeben.
%% Das Zugriffsdatum (urldate) spielt hier keine Rolle.
%% Für Nicht-Online-Quellen wird nichts geändert.
\renewbibmacro*{date+extradate}{%
  \printtext[parens]{%
    \printfield{usera}%
    \setunit{\printdelim{titleyeardelim}}%
    \ifentrytype{online}
       {\setunit*{\addspace\addcomma\addspace}%
         \iffieldundef{year}
           {\bibstring{nodate}}
       {\printlabeldateextra}}%
       {\printlabeldateextra}}}

%% Anzeige des Jahres nach dem Stichwort (usera) in der Fussnote
%% das Stichwort hat der Aufrufer hier schon ausgegeben.
%% siehe auch Kommentar zu: \renewbibmacro*{date+extradate}
\renewbibmacro*{cite:labeldate+extradate}{%
    \ifentrytype{online}
       {\setunit*{\addspace\addcomma\addspace}%
         \iffieldundef{year}
           {\bibstring{nodate}}
       {\printlabeldateextra}}%
       {\printlabeldateextra}}


\DefineBibliographyStrings{german}{
  nodate    = {{}o.\adddot\addspace J\adddot},
  andothers = {et\addabbrvspace al\adddot}
}
\DefineBibliographyStrings{english}{
  nodate    = {{}n.\adddot\addspace d\adddot},
  andothers = {et\addabbrvspace al\adddot}
}
\DeclareSourcemap{
  \maps[datatype=bibtex]{
    \map{
      \step[notfield=translator, final]
      \step[notfield=editor, final]
      \step[fieldset=author, fieldvalue={{{\langde{o\noexpand\adddot\addspace V\noexpand\adddot}\langen{Anon}}}}]
    }
    \map{
      \pernottype{online}
      \step[fieldset=location, fieldvalue={\langde{o\noexpand\adddot\addspace O\noexpand\adddot}\langen{s\noexpand\adddot I\noexpand\adddot}}]
    }
  }
}

\renewbibmacro*{cite}{%
  \iffieldundef{shorthand}
    {\ifthenelse{\ifnameundef{labelname}\OR\iffieldundef{labelyear}}
       {\usebibmacro{cite:label}%
        \setunit{\printdelim{nonametitledelim}}}
       {\printnames{labelname}%
        \setunit{\printdelim{nametitledelim}}}%
     \printfield{usera}%
     \setunit{\printdelim{titleyeardelim}}%
     \usebibmacro{cite:labeldate+extradate}}
    {\usebibmacro{cite:shorthand}}}

    \renewcommand*{\jourvoldelim}{\addcomma\addspace}% Trennung zwischen journalname und Volume. Sonst Space; Laut Leitfaden richtig
    %Aufgrund der Änderung bzgl des Issues 169 in der thesis_main.tex musste ich die Zeile auskommentieren. Konnte aber das Verhalten, dass die Fußnoten grün sind, im nachhinein nicht feststellen.
    %\hypersetup{hidelinks} %sonst sind Fußnoten grün. Dadurch werden Links allerdings nicht mehr farbig dargestellt

\renewbibmacro*{journal+issuetitle}{%
  \usebibmacro{journal}%
  \setunit*{\jourvoldelim}%
  \iffieldundef{series}
    {}
    {\setunit*{\jourserdelim}%
     \printfield{series}%
     \setunit{\servoldelim}}%
  \iffieldundef{volume}
    {}
    {\printfield{volume}}
  \iffieldundef{labelyear}
  {}
  {
  (\thefield{year}) %Ansonsten wird wenn kein Volume angegeben ist ein Komma vorangestellt
  }
  \setunit*{\addcomma\addspace Nr\adddot\addspace}
  \printfield{number}
  \iffieldundef{eid}
  {}
  {\printfield{eid}}
}

% Postnote ist der Text in der zweiten eckigen Klammer bei einem Zitat
% wenn es keinen solchen Eintrag gibt, dann auch nicht ausgeben, z.B. 'o. S.'
% Wenn man das will, kann man das 'o. S.' ja explizit angeben. Andernfalls steht
% sonst auch bei Webseiten 'o. S.' da, was laut Leitfaden nicht ok ist.
\renewbibmacro*{postnote}{%
  \setunit{\postnotedelim}%
  \iffieldundef{postnote}
    {} %{\printtext{\langde{o.S\adddot}\langen{no page number}}}
    {\printfield{postnote}}}

% Abstand bei Änderung Anfangsbuchstabe ca. 1.5 Zeilen
\setlength{\bibinitsep}{0.75cm}

% nur in den Zitaten/Fussnoten den Vornamen abkürzen (nicht im
% Literaturverzeichnis)

\DeclareDelimFormat{nonameyeardelim}{\addcomma\space}
\DeclareDelimFormat{nameyeardelim}{\addcomma\space}

\renewbibmacro*{cite}{%
  \iffieldundef{shorthand}
    {\ifthenelse{\ifciteibid\AND\NOT\iffirstonpage}
       {\usebibmacro{cite:ibid}}
    {\printtext[bibhyperref]{\ifthenelse{\ifnameundef{labelname}\OR\iffieldundef{labelyear}}
       {\usebibmacro{cite:label}%
        \setunit{\printdelim{nonameyeardelim}}}
      {\toggletrue{abx@bool@giveninits}%
        \printnames[family-given]{labelname}%
        \setunit{\printdelim{nameyeardelim}}}%
      \printfield{usera}%
      \setunit{\printdelim{titleyeardelim}}%
     \usebibmacro{cite:labeldate+extradate}}}}
   {\usebibmacro{cite:shorthand}}}