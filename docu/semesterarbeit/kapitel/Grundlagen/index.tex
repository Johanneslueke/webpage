\newpage

\section{Methodik} 
Das Projekt wird agil und iterativ entwickelt. Es wird keine spezifische Entwicklungsmethodik strikt angewandt, da das Projekt als 
Einzelarbeit durchgeführt wird. Dennoch werden Elemente aus dem Scrum-Prozess genutzt, insbesondere die Scrum-Artefakte. Die traditionellen 
Scrum-Rollen und -Zeremonien werden in diesem Projekt weggelassen, da diese hauptsächlich in Teams ab drei Personen sinnvoll sind, was auf 
persönlichen Erfahrungen aus dem beruflichen Umfeld basiert.

Die agile Methodik ist für dieses Projekt besonders geeignet, da sie Flexibilität und Anpassungsfähigkeit während des Entwicklungsprozesses bietet. 
Durch iterative Entwicklungszyklen können Anforderungen und Lösungen kontinuierlich weiterentwickelt und verbessert werden.

\subsection{ Scrum/Kanban }

Scrum ist ein bewährter agiler Ansatz, der auf festen Zeiträumen (Sprints) basiert, in denen das Team spezifische Aufgaben abschließt. 
Obwohl die volle Implementierung von Scrum für dieses Einzelprojekt nicht notwendig ist, werden bestimmte Elemente übernommen:

\begin{itemize}
    \item Scrum-Artefakte: Diese umfassen das Product Backlog, Sprint Backlog und die Increment. Diese Artefakte helfen dabei, die anstehenden 
    Aufgaben zu organisieren und den Fortschritt zu verfolgen.
\end{itemize}

Kanban ist ein weiteres agiles Framework, das ein visuelles Board zur Organisation von Aufgaben verwendet. Die Aufgaben werden in Spalten 
wie \"Offen\", \"In Arbeit\" und \"Erledigt\" angezeigt. Kanban ist besonders flexibel und kann leicht an verschiedene Arbeitsumgebungen angepasst werden.

\begin{itemize}
    \item Visuelles Board: Durch die Visualisierung des Arbeitsflusses können Fortschritte und Engpässe leicht erkannt und entsprechend reagiert werden.
    \item Flexibilität: Kanban ermöglicht es, die Anzahl der Aufgaben in jeder Spalte anzupassen und kontinuierlich Verbesserungen vorzunehmen.
\end{itemize}

Scrumban kombiniert die Struktur von Scrum mit der Flexibilität von Kanban und bietet somit das Beste aus beiden Welten. Dieser Ansatz ermöglicht es, 
die Aufgaben effizient zu organisieren und gleichzeitig flexibel auf Veränderungen zu reagieren.

Für dieses Projekt wird eine angepasste Version von Scrumban angewendet:
\begin{itemize}
    \item Aufgabenmanagement: Die Aufgaben werden auf einem Kanban-Board organisiert und verfolgt. Dies ermöglicht eine klare Visualisierung des Arbeitsfortschritts.
    \item Iterative Entwicklung: Das Projekt wird in kleinen, überschaubaren Iterationen entwickelt. Dies ermöglicht eine kontinuierliche Überprüfung und Anpassung der Projektanforderungen und -lösungen.
    \item Flexibilität: Durch die Kombination der besten Praktiken aus Scrum und Kanban wird eine flexible und adaptive Arbeitsweise sichergestellt, die es ermöglicht, auf Veränderungen schnell zu reagieren und den Entwicklungsprozess kontinuierlich zu verbessern. 
\end{itemize}

Durch die Anwendung dieser Methodik wird sichergestellt, dass das Projekt effizient und effektiv durchgeführt wird, und gleichzeitig die Flexibilität und Anpassungsfähigkeit während des gesamten Entwicklungsprozesses gewahrt bleibt.

\section{Theorie} 


\subsection{ User Experiance }

Der Begriff User Experience (UX) beschreibt alle Aspekte der Eindrücke und das Erlebnis eines Nutzers bei der Interaktion mit einem Produkt, 
einer Dienstleistung, einer Umgebung oder einer Einrichtung. Dabei geht es nicht nur um digitale Produkte wie Websites oder Apps, sondern auch 
um physische Nutzung.

Die User Experience umfasst verschiedene Kriterien:

\begin{itemize}
    \item Nützlichkeit (Usability): Ein Produkt sollte seine Funktionen einfach zugänglich machen und die gewünschte Funktionalität bieten.
    \item Festigkeit (Stabilität): Produkte sollten zuverlässig sein und nicht zusammenbrechen, abstürzen oder reparaturanfällig sein.
    \item Ästhetik und Emotionalität: Schönheit und das ganzheitliche Erlebnis spielen ebenfalls eine Rolle. Ästhetik ist jedoch nicht für jedes Produkt gleich wichtig.

\end{itemize}

In der Produktentwicklung ist die User Experience ein zentraler Faktor, um sicherzustellen, dass Nutzer Produkte gerne verwenden und ihre Erwartungen 
erfüllt werden. Berufsfelder wie UX Researcher, UX Designer und UX Writer beschäftigen sich intensiv mit diesem Thema.

Die User Experience (UX) spielt eine zentrale Rolle in der Softwareentwicklung. Sie bildet den Rahmen für eine am Nutzer ausgerichtete Gestaltung von .
Produkten und Anwendungen. Hier sind einige Aspekte, die zeigen, warum UX so wichtig ist:

\begin{itemize}
    \item Frühe Problemerkennung: Durch UX-Design können Probleme bereits in der Entwicklungsphase entdeckt werden. Das verhindert teure Korrekturen später im Prozess.
    \item Nutzerzentrierter Ansatz: UX-Designer:innen nehmen die Perspektive der Nutzer:innen ein. Sie sorgen für frustfreie, leicht erlernbare und intuitiv zu bedienende Produkte.
    \item Qualitätssicherung: UX-Design beeinflusst die Produktentwicklung von der ersten Vision bis zur finalen Auslieferung. Es hilft sicherzustellen, dass die Anwendung den Erwartungen der Nutzer:innen entspricht.

\end{itemize}

Insgesamt trägt eine gute User Experience dazu bei, dass Produkte erfolgreich sind und Nutzer:innen langfristig begeistern
 
\subsection{ Single Page Application }
Eine Single Page Application (SPA) ist eine moderne Art der Webanwendung, die nur aus einer einzigen Webseite besteht. Im 
Gegensatz zu traditionellen Multi-Page-Anwendungen, die aus vielen separaten HTML-Dokumenten bestehen, lädt eine SPA nur ein 
einziges HTML-Dokument. Dieses wird durch JavaScript-APIs wie Fetch dynamisch aktualisiert, wenn unterschiedliche Inhalte 
angezeigt werden sollen.

Die Funktionsweise einer SPA basiert auf einem initialen HTML-Dokument, das neben dem grundlegenden Aufbau und Design auch ein
DOM-Element enthält. Dieses Document Object Model wird im Hintergrund durch JavaScript-Code ständig manipuliert, um die 
Funktionalität der Website sicherzustellen. Bei Nutzerinteraktionen werden Daten im JSON- oder XML-Format geladen und 
automatisch in das DOM der geladenen Webseite eingefügt. Dadurch arbeitet die gesamte Präsentationslogik direkt im Browser, 
ohne dass die Website ständig neu geladen werden muss.


\subsection{CI/CD}
Continuous Integration (CI) bezieht sich auf die Praktik, Codeänderungen automatisch und regelmäßig in ein gemeinsames 
Quellcode-Repository zu integrieren. Dabei werden automatisierte Testschritte ausgelöst, um die Zuverlässigkeit der zusammengeführten 
Codeänderungen sicherzustellen.

Continuous Delivery und/oder Continuous Deployment (CD) sind zweiteilige Prozesse, die die Integration, das Testen und die Bereitstellung
der Codeänderungen umfassen. Continuous Delivery beinhaltet kein automatisches Produktiv-Deployment, während beim Continuous Deployment 
Update-Releases automatisch in die Produktivumgebung übergeben werden. Diese zusammenhängenden Praktiken werden oft als “CI/CD-Pipeline”
bezeichnet und unterstützen Entwicklungs- und Operations-Teams bei der agilen Zusammenarbeit mit einem DevOps- oder Site-Reliability-Engineering-Ansatz

\section{Technologien}




\subsection{ Angular }
Angular ist ein Web-Framework, das Entwicklern ermöglicht, schnelle und zuverlässige Anwendungen zu erstellen. Es wird von 
einem engagierten Team bei Google gepflegt und bietet eine breite Palette von Tools, APIs und Bibliotheken, um Ihren 
Entwicklungsworkflow zu vereinfachen und zu optimieren.

Angular basiert auf TypeScript und ermöglicht den Aufbau effizienter und anspruchsvoller Single-Page-Anwendungen. Es bietet 
eine Komponentenarchitektur, die die Strukturierung von Projekten in gut organisierte Teile mit klaren Verantwortlichkeiten 
erleichtert. Jede Angular-Komponente besteht aus einem Dekorator, einem HTML-Template und einer TypeScript-Klasse, die das 
Verhalten der Komponente definiert.
 
\subsection{ Vercel }
Vercel ist eine Plattform für Frontend-Entwickler, die die Geschwindigkeit und Zuverlässigkeit bietet, die Innovatoren 
benötigen, um im Moment der Inspiration zu erstellen. Sie ermöglicht es Entwicklern, Webanwendungen einfach bereitzustellen 
und die Benutzeroberfläche ihrer Anwendungen getrennt von der Backend-Logik zu betreiben. Vercel bietet Funktionen wie 
Deploy-Previews, Functions as a Service, Analysen und mehr. Darüber hinaus hat Vercel cloudähnliche Features wie Speicher 
und Datenbanken hinzugefügt.

\begin{itemize}
    \item Bereitstellung und Hosting: Vercel ermöglicht es Entwicklern, ihre Webanwendungen einfach zu bereitstellen. Du kannst deine Anwendung direkt aus deinem Git-Repository (z. B. GitHub oder GitLab) auf Vercel hosten. Die Plattform kümmert sich um die Bereitstellung, Skalierung und Wartung der Infrastruktur.
    \item Serverless Functions: Vercel bietet eine Funktion als Dienst (FaaS) namens “Serverless Functions”. Damit kannst du serverseitige Logik in deiner Anwendung ausführen, ohne dich um die Verwaltung von Servern kümmern zu müssen. Diese Funktionen sind skalierbar und werden bei Bedarf automatisch gestartet.
    \item Deploy-Previews: Wenn du Änderungen an deinem Code vornimmst und einen Pull Request erstellst, generiert Vercel automatisch eine “Deploy-Preview”. Das ist eine temporäre Version deiner Anwendung, die du testen kannst, bevor du sie in die Produktion überführst.
    \item Performance und Zuverlässigkeit: Vercel optimiert automatisch deine Anwendung für Geschwindigkeit und Zuverlässigkeit. Dazu gehören Funktionen wie automatische Code-Splitting, Caching und globale Content Delivery Networks (CDNs).
    \item Analytics und Monitoring: Du kannst die Leistung deiner Anwendung mit Vercel-Analysen überwachen. Das hilft dir, Engpässe zu erkennen und Verbesserungen vorzunehmen.
\end{itemize}

\subsection{ Git }

Git ist ein Versionskontrollsystem, das Änderungen an Dateien verfolgt und die Zusammenarbeit mehrerer Personen an denselben 
Projekten ermöglicht. Es hilft, den Code im Laufe der Zeit zu verwalten und ermöglicht es, zu früheren Versionen zurückzukehren,
 wenn etwas schiefgeht. Git ist besonders nützlich, um gleichzeitige Arbeiten von verschiedenen Teammitgliedern zu koordinieren,
  ohne dass ihre Änderungen kollidieren. Wenn du ein Projekt klonst, lädst du eine Kopie davon auf deinen lokalen Computer 
  herunter. Dann kannst du verschiedene Befehle verwenden, um Änderungen zu verfolgen, Branches zu erstellen, Code zu 
  überprüfen und deine Arbeit mit anderen zu synchronisieren.

\subsection{ Markdown }

Markdown ist eine textbasierte Auszeichnungssprache, die von John Gruber entwickelt wurde. Sie ermöglicht es, Text für das Web einfach 
zu schreiben und zu formatieren. Dabei handelt es sich um eine leicht lesbare und schreibbare Syntax, die dann in strukturell gültiges 
XHTML (oder HTML) umgewandelt wird.

Mit Markdown kannst du Texte in verschiedenen Formaten wie HTML, PDF und mehr erstellen. Es ist besonders bei Entwicklern beliebt, 
da es fein abgestimmte Kontrolle über Text und Code bietet

\begin{itemize}
    \item Überschriften: Du kannst Überschriften mit Hashtags erstellen, z. B. \# Überschrift der ersten Ebene. Markdown unterstützt auch mehrere Ebenen von Überschriften.
    \item Textformatierung: Du kannst Texte kursiv, fett oder beides formatieren. Zum Beispiel: *kursiver Text*, **fetter Text** oder ***fetter und kursiver Text***.
    \item Listen: Erstelle ungeordnete Listen mit Bindestrichen (-) oder geordnete Listen mit Zahlen (., ., usw.).
    \item Links und Bilder: Verlinke Text mit Text und füge Bilder mit !Alternativtext ein.
    \item Codeblöcke: Um Code zu kennzeichnen, verwende einfache Backticks (\`) für einzelne Wörter oder dreifache Backticks (```) für mehrzeiligen Code.
\end{itemize}
