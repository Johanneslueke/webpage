\section{Fazit}
Das Projekt zur Konzeptionierung, Gestaltung und Implementierung einer Webseite zur Selbstdarstellung und zum Veröffentlichen von Artikeln (Blog) 
hat erfolgreich eine Software geschaffen, die sowohl für die persönliche Präsentation als auch für das Teilen von Fachwissen genutzt werden kann. 
Durch den Einsatz von modernen Webtechnologien wie Angular, Vercel und Git, kombiniert mit einer agilen Methodik, konnte eine stabile und 
benutzerfreundliche Webseite entwickelt werden. Die Umsetzung der Webseite beinhaltete die Implementierung von CI/CD Best Practices, 
die sicherstellen, dass die Seite kontinuierlich aktualisiert und verbessert werden kann, ohne die Benutzererfahrung zu beeinträchtigen.

Die Webseite dient als digitale Visitenkarte und zeigt die technischen Fähigkeiten und praktischen Erfahrungen des Entwicklers. 
Das Blogging-System ermöglicht es, regelmäßig neue Inhalte zu veröffentlichen, ohne zusätzlichen Programmieraufwand. 
Insgesamt wurden die gesetzten Ziele erreicht und eine solide Grundlage für zukünftige Erweiterungen und Anpassungen geschaffen.

\textbf{Ausblick auf noch offene Arbeiten}
Trotz der erfolgreichen Umsetzung des Projekts gibt es einige Bereiche, die weiter verbessert und ausgebaut werden können:

Erweiterung der Blog-Funktionalitäten: Es könnten weitere Features wie eine erweiterte Suchfunktion, Kategorisierungen und Tagging-Systeme implementiert werden, um die Benutzerfreundlichkeit zu erhöhen und die Navigation zu erleichtern.

\begin{itemize}
    \item \textbf{SEO-Optimierung}: Eine umfassende Suchmaschinenoptimierung (SEO) könnte die Sichtbarkeit der Webseite in Suchmaschinen verbessern und mehr Besucher anziehen.

    \item \textbf{Integration von Analytics}: Durch die Integration von Analysetools können detaillierte Einblicke in das Nutzerverhalten gewonnen und die Webseite entsprechend optimiert werden.
    
    \item \textbf{Erweiterung des Portfolios}: Das Portfolio kann kontinuierlich mit neuen Projekten und Erfahrungen erweitert werden, um den aktuellen Stand der Fähigkeiten und Kenntnisse des Entwicklers widerzuspiegeln.
    
    \item \textbf{Performance-Optimierungen}: Obwohl die Webseite bereits gut funktioniert, können weitere Performance-Optimierungen vorgenommen werden, um die Ladezeiten weiter zu verkürzen und die Benutzererfahrung zu verbessern.
    
    \item \textbf{Barrierefreiheit}: Eine umfassende Überprüfung und Verbesserung der Barrierefreiheit der Webseite könnte vorgenommen werden, um sicherzustellen, dass sie für alle Benutzer, einschließlich Menschen mit Behinderungen, zugänglich ist.
    
\end{itemize}
Durch die kontinuierliche Arbeit an diesen offenen Punkten kann die Webseite nicht nur technisch, sondern auch inhaltlich und funktional weiterentwickelt werden, um den sich wandelnden Anforderungen und Erwartungen des Projektes gerecht zu werden.