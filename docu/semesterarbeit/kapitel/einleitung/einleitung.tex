\section{Kurzbeschreibung}
Das vorliegende Projekt zielt darauf ab, eine Webseite zu entwickeln, die als digitale Visitenkarte für den Entwickler dient und gleichzeitig eine Plattform für das Veröffentlichen von Artikeln bereitstellt. 
Die Umsetzung der Webseite ist Teil der Projektarbeiten im Studiengang Informatik an der FOM Hochschule für Oekonomie \& Management.

In der heutigen digitalen Welt ist eine professionelle Online-Präsenz für Entwickler und andere Fachkräfte unerlässlich. Eine gut gestaltete 
Webseite kann nicht nur als Portfolio dienen, sondern auch potenzielle Arbeitgeber und Auftraggeber von den eigenen Fähigkeiten überzeugen. Darüber hinaus 
bietet die Integration eines Blogging-Systems die Möglichkeit,regelmäßig Inhalte zu veröffentlichen und Fachwissen zu teilen, ohne zusätzlichen Programmieraufwand.

\textbf{ Zusammenfassung } \\
Das Projekt dient als persönliche Visitenkarte und Showcase für die Fähigkeiten im Bereich Webentwicklung. Durch die Implementierung moderner 
Webtechnologien und die Anwendung agiler Entwicklungsmethoden wird eine stabile und benutzerfreundliche Webseite geschaffen.
Diese bietet nicht nur eine professionelle Darstellung des Entwicklers, sondern auch eine Plattform für den Austausch von Wissen und Erfahrungen.

\textbf{ Ergebniss } \\
Vollständig Implementiere Webseite inclusive CI/CD

\section{Ziele}
Die Webseite verfolgt drei Hauptziele:

\begin{itemize*}
    \item Continuous Integration \& Continuous Delivery (CI/CD): Sicherstellung eines kontinuierlich laufenden Systems, das bei jeder Änderung automatisch aktualisiert wird.
    \item Blogging-System: Bereitstellung einer Plattform zum Veröffentlichen von Artikeln, die unabhängig vom Implementierungscode verwaltet werden kann.
    \item Portfolio: Präsentation der praktischen Erfahrungen und Fähigkeiten des Entwicklers in einem ansprechenden Format.

\end{itemize*}

Das erste Ziel verfolgt den wunsch zu jedem Zeitpunkt und mit jeder Änderung stehts ein laufendes
System bereitzustellen. Jeder Besucher solle stehts in der Lage sein die Webseite zu besuchen und
Inhalt angzeigt zu bekommen.
Das Zweite Ziel hat zum Thema, die Bereitstellung von selbst geschriebenen Artikeln auf der Webseite
zu ermöglichen ohne neuen programmieraufwand aufwänden zu müssen. Die Navigation zu neuen Artikeln
sollte dynamisch erfolgen und der Inhal entkoppelt vom Code angzeigt werden können.
Das Dritte und Letzte Ziel des Projekts, ist es ein Portfolio bereitzustellen welches zeigt was der
Entwickler an Praktischen Erfahrungen gesammelt hat in seinem Berufliche und Privaten Umfeld. Dies
schließt ein CV und Resumee über den Entwickler mit ein.


\subsection{ Aufstellung der einzelnen Ziele }

\textbf{Continuous Integration \& Continuous Delivery (CI/CD)}

Ziel: Sicherstellung eines kontinuierlich laufenden Systems, das bei jeder Änderung automatisch aktualisiert wird.
\begin{itemize}
    \item Implementierung einer CI/CD-Pipeline.
    \item Automatische Tests und Deployments bei Codeänderungen.
    \item Gewährleistung der Systemstabilität und -verfügbarkeit.
\end{itemize}

\textbf{Blogging-System}

Ziel: Bereitstellung einer Plattform zum Veröffentlichen von Artikeln, die unabhängig vom Implementierungscode verwaltet werden kann.
\begin{itemize}
    \item Integration eines Markdown-Editors zur einfachen Erstellung und Verwaltung von Blogbeiträgen.
    \item Dynamische Anzeige und Filterung der Artikel.
    \item Unterstützung für Code-Snippet-Highlighting und Meta-Daten-Anzeige.
    \item Möglichkeit, Blog-Einträge mit Cover-Bildern hervorzuheben.
\end{itemize}

\textbf{Portfolio}

Ziel: Präsentation der praktischen Erfahrungen und Fähigkeiten des Entwicklers in einem ansprechenden Format.
\begin{itemize}
    \item Erstellung eines übersichtlichen Lebenslaufs (CV).
    \item Darstellung von Projekten und praktischen Erfahrungen.
    \item Pflege und Aktualisierung der Portfolio-Inhalte ohne großen Aufwand.
    \item Bereitstellung von Kontaktinformationen und Links zu sozialen Medien oder anderen relevanten Plattformen.
\end{itemize}