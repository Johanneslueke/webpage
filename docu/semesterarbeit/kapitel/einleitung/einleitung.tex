\section{Kurzbeschreibung}


\textbf{ Zusammenfassung } \\
Das Projekt befasst sich mit der Umsetzung einer Webseite. Inhalt der Webseite ist ein Resuméé 
und Lebenslauf zur selbstrepräsentation gegenüber Potentiellen Arbeitgebern und Arbeitnehmern.
Zusäzlich soll ein Blogging-System mit in die Webseite integiert werden, dessen initalien Inhalt die Entwicklung der
Webseite selbs beschreibt.

Blogging Content soll unabhängig von dem Implemenentierung der Webseite sein. Inhalte werden über
das bekannte Markdown-Format bereitgestellt.

\textbf{ Ergebniss } \\
Vollständig Implementiere Webseite inclusive CI/CD

\section{Ziele}
Das Projekt verfolgt 3 große Ziele. Diese lauten: 
    \begin{itemize}
        \item Continous Integration \& Continous Delivery wird erfolgreich angewandt
        \item Blogging
        \item Portfolio 
    \end{itemize}

Das erste Ziel verfolgt den wunsch zu jedem Zeitpunkt und mit jeder Änderung stehts ein laufendes
System bereitzustellen. Jeder Besucher solle stehts in der Lage sein die Webseite zu besuchen und
Inhalt angzeigt zu bekommen.
Das Zweite Ziel hat zum Thema, die Bereitstellung von selbst geschriebenen Artikeln auf der Webseite
zu ermöglichen ohne neuen programmieraufwand aufwänden zu müssen. Die Navigation zu neuen Artikeln
sollte dynamisch erfolgen und der Inhal entkoppelt vom Code angzeigt werden können.
Das Dritte und Letzte Ziel des Projekts, ist es ein Portfolio bereitzustellen welches zeigt was der
Entwickler an Praktischen Erfahrungen gesammelt hat in seinem Berufliche und Privaten Umfeld. Dies
schließt ein CV und Resumee über den Entwickler mit ein.

\subsection{ Projektbegründung }
Das Projekt dient Als Visitenkarte für den Entwickler und ' Show Cases ' seine Fähigkeiten im Bereich
des Webdevelopements.

\subsection{ Aufstellung der einzelnen Ziele }
